% !TEX program = xelatex

\documentclass{resume}
\usepackage{graphicx}
\usepackage{tabularx}
\usepackage{tabu}
\usepackage{multirow}
\usepackage{progressbar}
% \usepackage{zh_CN-Adobefonts_external} % Simplified Chinese Support using external fonts (./fonts/zh_CN-Adobe/)
\usepackage{zh_CN-Adobefonts_internal} % Simplified Chinese Support using system fonts

\begin{document}
\pagenumbering{gobble} % suppress displaying page number

% {
% % change Large font here
% \Large{
%   \begin{tabu}{ c l r }
%    \multirow{5}{1in}{\includegraphics[width=0.88in]{avatar}} & \scshape{Bin Yuan} & {Python~}\progressbar{0.75} \\
%     & \email{yuanbin2014@gmail.com} & {Scala~}\progressbar{0.5} \\
%     & \phone{(+86) 131-221-87xxx} & {Linux~}\progressbar{0.7} \\
%     & \linkedin[billryan8]{https://www.linkedin.com/in/billryan8} & {Flask~}\progressbar{0.5} \\
%     & \github[github.com/billryan]{https://github.com/billryan} & {Javascript~}\progressbar{0.5}
%   \end{tabu}
% }
% }

% {
% % change Large font here
% \Large{
%   \begin{tabu}{ c l }
%    \multirow{4}{1in}{\includegraphics[width=0.88in]{lzj_avatar}} & \scshape{Zujing Liu} \\
%     & \email{zujing.liu@whu.edu.cn} \\
%     & \phone{(+86) 187-9005-5889} \\
%     & \homepage[liuxiaozhu.github.io]{https://liuxiaozhu01.github.io/} \\
%     & \github[github.com/liuxiaozhu01]{https://github.com/liuxiaozhu01} 
%   \end{tabu}
% }
% }

{
% change Large font here
\Large{
  \begin{tabularx}{\textwidth}{@{} c X r @{}}
   \multirow{4}{1in}[1.em]{\includegraphics[width=0.88in]{lzj_avatar}} & \textbf{刘祖靖} & \email{zujing.liu@whu.edu.cn} \\
    & 武汉大学\ 计算机学院 & \phone{(+86) 187-9005-5889} \\
    & 计算机科学与技术 & \homepage[liuxiaozhu.github.io]{https://liuxiaozhu01.github.io/} \\
    &                  & \github[github.com/liuxiaozhu01]{https://github.com/liuxiaozhu01} 
  \end{tabularx}
}
}

\section{\faGraduationCap\ 教育经历}
\datedsubsection{\textbf{武汉大学}}{2023.09 -- 至今}
\textit{学术型硕士(推免)}————计算机科学与技术————导师夏桂松(杰青,教授)、高源(副教授)
% \datedsubsection{计算机学院-计算机科学与技术-导师夏桂松(杰青,教授)、高源(副教授)}{\textit{学术型硕士(推免)}}
\datedsubsection{\textbf{武汉大学}}{2019.09 -- 2023.07}
\textit{学士(荣誉学位)} ————计算机科学与技术(弘毅班)————GPA3.81/4.00, Top14\% 
% \datedsubsection{计算机学院-计算机科学与技术(弘毅班)-GPA3.81/4.00, Top14\% }{\textit{学士(荣誉学位)}}

\section{\faTasks\ 项目经历}
\datedsubsection{\textbf{智能可信辨识技术及在城市安全领域的应用}}{2020.09 -- 2021.08(本科)}
% 湖北省技术创新专项(重大项目)
\begin{itemize}
  \item 开发室内场景 3D 行人仿真模块,实现实时渲染与动态目标追踪,支持自由视点交互操作。
  % 实现真实室内情景实时模拟,提供自由视角与对象跟踪等可操作功能。
  \item 主导视频展示模块,设计实现 “流水线式视频处理分析框架”,视频处理分析延迟降低60\%。
  \item 提出 “视频流帧率调整方法、装置、设备及可读存储介质”,现已获得发明专利授权。% 解决视频帧率波动与缺失帧问题,(专利号:ZL 2021 1 1639965.1)
\end{itemize}


\section{\faUsers\ 科研经历}
\datedsubsection{\textbf{Bypass Back-propagation: Optimization-based Structural Pruning for Large Language Models via Policy Gradient}}{ACL 投稿}
% \role{Summer Intern}{Manager: xxx}
% Brief introduction: xxx.
\begin{itemize}
  \item 一种\textcolor{red}{基于优化的 LLM 结构化剪枝方法},直接以模型损失为剪枝优化目标。
  \item 使用伯努利分布对剪枝掩码进行概率化建模,通过策略梯度估计器优化概率参数,\textbf{避免反向传播}。
  \item 与现有的启发式剪枝技术相比,性能更优越,并兼顾计算高效性。
\end{itemize}

\section{\faTrophy\ 荣誉奖项}
\datedsubsection{本科阶段}{2019 -- 2023}
\begin{itemize}[parsep=0.5ex]
  \item 武汉大学乙等学业奖学金(2021)
  \item 武汉大学三好学生(2021)
  \item 国家励志奖学金(2020-2022)
\end{itemize}

%% Reference
%\newpage
%\bibliographystyle{IEEETran}
%\bibliography{mycite}
\end{document}
